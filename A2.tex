\documentclass[11pt, answers]{exam}
\renewcommand{\baselinestretch}{1.05}
\usepackage{amsmath,amsthm,verbatim,amssymb,amsfonts,amscd, graphicx}
\usepackage{graphics}

\usepackage{afterpage}
\usepackage{caption}

\usepackage{fancybox}

\usepackage{clrscode3e}
\usepackage{amsmath}
\usepackage{amssymb}

\topmargin0.0cm
\headheight0.0cm
\headsep0.0cm
\oddsidemargin0.0cm
\textheight23.0cm
\textwidth16.5cm
\footskip1.0cm
\theoremstyle{plain}
\theoremstyle{definition}
\newtheorem{theorem}{Theorem}
\newtheorem{corollary}{Corollary}
\newtheorem{lemma}{Lemma}
\newtheorem{proposition}{Proposition}
\newtheorem*{surfacecor}{Corollary 1}
\newtheorem{conjecture}{Conjecture}  
\newtheorem{definition}{Definition}
\newcommand{\R}{\mathbb{R}}

\title{CSC336: Assignment 2}
\author{Junjie Cheng, 1002770539}
\date{November 1 2017}

\begin{document}

\maketitle

\begin{questions}
\question %Q1
Let $x$ be an arbitrary number in $\R^n$ for arbitrary $n \ge 1$, by definition, $||x||_2 = \sqrt{\Sigma_{i=1}^n x_i^2} $. 

By customary we define $\Sigma_{k=n}^m$ with $n>m$ as the empty sum (i.e. $0$). Then we have

$||x||_1 = \Sigma_{i=1}^n |x_i| = \sqrt{(\Sigma_{i=1}^n |x_i|)^2} = \sqrt{\Sigma_{i=1}^n x_i^2 + 2\Sigma_{i=1}^n \Sigma_{j=i+1}^n |x_i||x_j|}$.

Since $|x_i| \ge 0$ for all $1 \le i \le n$, $|x_i||x_j| \ge 0$ for all $1 \le i \le j \le n$. So, $2\Sigma_{i=1}^n \Sigma_{j=i+1}^n |x_i||x_j| \ge 0$ for all $n \ge 1$.

Then we have $||x||_2 = \sqrt{\Sigma_{i=1}^n x_i^2} \le \sqrt{\Sigma_{i=1}^n x_i^2 + 2\Sigma_{i=1}^n \Sigma_{j=i+1}^n |x_i||x_j|} = ||x||_1$.

\question %Q2
It is possible. Here is an example.

$x = (48, 55)^T, ||x||_1 = |48|+|55|= 103, ||x||_2 = \sqrt{48^2+55^2} = 73$;

$y = (13, 84)^T, ||y||_1 = |13|+|84|= 97,  ||y||_2 = \sqrt{13^2+84^2} = 85$;

$||x||_1 = 103 > 97 = ||y||_1$ while $||x||_2 = 73 < 85 = ||y||_2$.

\question %Q3
Since $Ax = b$ and $A\hat{x} = \hat{b}$, we have
$A(x-\hat{x}) = b - \hat{b}$. 

So $||b-\hat{b}|| = ||A(x-\hat{x})|| \le ||A||\cdot||x-\hat{x}||$.

On the other hand, $A^{-1}b = x$, so $||x|| = ||A^{-1}b|| \le ||A^{-1}||\cdot||b||$.

Since all norms are equal or greater than zero, we have $||b-\hat{b}|| \cdot ||x|| \le ||A||\cdot||A^{-1}|| \cdot ||x-\hat{x}|| \cdot ||b||$, so $\frac{1}{||A||\cdot||A^{-1}||}\frac{||b-\hat{b}||}{||b||} \le \frac{||x-\hat{x}||}{||x||}$, i.e. $ \frac{1}{cond(A)}\frac{||b-\hat{b}||}{||b||} \le \frac{||x-\hat{x}||}{||x||}$.
\end{questions}
\end{document}
